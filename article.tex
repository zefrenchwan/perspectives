\documentclass{article}

\title{Moteur}
\author{zefrenchwan}
\date{version du \today{}}

\usepackage[french]{babel}
\usepackage{amsfonts,amsmath,amssymb}

\begin{document}
\maketitle
\abstract{Cet article se concentre sur une description d'un espace d'états. Dans un premier temps, un modèle mathématique général est proposé. Au lieu d'observer un objet évoluer dans un repère, on définit une trajectoire dans un espace d'états. La version graphe permet une meilleure gestion des objets observés, avec la prise en compte d'une mémoire. Les nœuds transforment des événements, et les liens les déplacent. Dès lors, on peut modéliser des systèmes sociaux, comme par exemple des réseaux sociaux. }
\section{Modèle de gestion d'états d'un système}
Prenons un exemple classique : des objets physiques sont représentés comme des masses ponctuelles. 
On se donne un repère $R$ dans un espace de dimension $N=3$ (hormis le temps).
Pour chacune, des forces de gravitation s'appliquent, soumettant chacun à une résultante $\vec{F}$. 
Dès lors, on applique la mécanique Newtonienne : $$\vec{F} = \frac{d\vec{P}}{dt}$$
On peut réaliser une simulation physique de la manière suivante, en fixant un pas $\Delta t > 0$ : 
\begin{enumerate}
\item définir un instant initial $t_0$
\item pour chaque élément $e$ du système, ajouter $e$ dans l'espace avec une position initiale et une quantité de mouvement 
\item pour chaque pas de temps : calculer la force appliquée sur $e$, formant une résultante $R_e$, appliquer $\Delta \vec{P}_e = \vec{R_e} \Delta t$, et finalement mettre à jour la position et la quantité de mouvement de l'élément $e$ 
\end{enumerate}

Si l'on reprend la formalisation précédente, on peut garder la même logique générale, disons l'algorithme. 
Cependant, on peut aussi écrire, dans un premier temps, $\Delta \vec{P}_e = I_N \vec{R_e} \Delta t$ avec $I_N$ la matrice identité. 
Le changement est mathématiquement équivalent, mais il est le prélude à une notion plus avancée : avoir un terme dépendant du point d'espace où est un élément. 

 
Formellement, on peut alors passer à quelque chose de plus général. 
On se donne un système physique qu'on décrit comme des objets $(o_i)_{i \in I}$ avec $I \neq \emptyset$. 
Chaque objet $o_i$ est décrit via un état $E_i$ variable dans le temps. 
Ainsi, au temps $t$, on a un vecteur $E_{it}$ dans un espace vectoriel de dimension $v$.
Pour ce moment $t$, et tout couple $(i,j) \in I^2$, on a un vecteur d'interaction $c_{ij}^t$ dans cet espace. 
Ainsi, pour un élément $o_i$ à un temps $t$, on a un ensemble de changements dont la résultante est $\mathcal{R}_{it}$ (qu'on avait noté $R_e$). 
Reste à expliquer la variation d'état $\Delta E_{it}$, et pas seulement comme un $\mathcal{R}_{it} \Delta t$.
C'est le rôle d'une transformation $M_{it}$ qui est pour le moment une matrice de taille $v \times v$. 
Elle s'interprète par analogie avec le tenseur métrique en relativité générale, qui définit la géométrie locale de l'espace et détermine comment une impulsion se traduit en mouvement selon la courbure de l'environnement.
On a alors $$\mathcal{R}_{it} \Delta t = M_{it} \Delta E_{it}$$
Cette forme est intéressante parce qu'elle donne une base pour arriver à une notion abstraite d'espace d'états. 
$$\text{}$$

\begin{tabular}{|l|l|l|}
	\hline 
	Concept & Cas classique & Généralisation \\
	\hline 
	\hline 
	Changement & Force & Tout élément vectorisable \\
	\hline 
	Résultante & Somme des forces & Fonction de réduction \\
	\hline 
	Fonction locale & Géométrie locale & Matrice d'états \\
	\hline 
	Indicateur & Quantité de mouvement & État vectorisé \\
	           & et position & \\
	\hline 
\end{tabular}
$$\text{}$$
Ainsi, le principe est de prendre un espace d'états, donc un espace vectoriel de dimension finie. 
Les changements peuvent être typés de n'importe quelle classe, tant que l'on sait produire une résultante sous forme de vecteur. 
Alors $\mathcal{R}_{it}$ est la résultante des changements subis par $o_i$ à $t$. 
Elle explique le changement d'état $\Delta E_{it}$. 
Pour lier les deux, la formule générale est $$R_{it} \Delta t = \Psi(E_{it}) \Delta E_{it}$$
mais le cas particulier le plus notoire reste une matrice $M_{it}$ ou un tenseur pour $\Psi$. 
C'est d'ailleurs cette forme là que nous utiliserons dans la suite. 
Rappelons qu'elle dépend du temps et de l'objet.
Les termes à gauche sont ce qui est relatif \`a la cause extérieure, les termes à droite sont ceux qui dépendent uniquement de l'état. Signalons au passage qu'il est tout à fait possible d'avoir une interaction réciproque de type : $$\Delta M_{it} = \Phi (\mathcal{R}_{it},E_{it}, \Delta t )$$
En terme physique, il est plus intéressant de le formuler en terme de $$E_{i,t+dt} = E_{it} + \Delta E_{it} = E_{it} + \Delta t M^{-1}_{it} R_{it} $$ mais cela implique des conditions sur $M_{it}$, évidemment qu'elle soit inversible pour commencer. 
\clearpage
En terme de "mode d'emploi" du modèle: 
\begin{enumerate}
	\item l'observateur perçoit un système dynamique composé des objets $(o_i)_{i \in I}$. Il sait observer, pour son temps $t$ à lui, les interactions de $o_i$ vers $o_j$ pour tout couple $(i,j) \in I$. 
	\item D'un point de vue formel, il connait donc $c_{ij}^t$ pour tout couple $(i,j) \in I^2$ et tout temps passé $t$. Chaque $c_{ij}^t$ est élément d'un ensemble $X$ dont la nature importe peu. 
	\item l'observateur modélise l'état des objets observés et les changements dans un même espace vectoriel de dimension $v$. Il dispose donc de fonctions de vectorisation. D'abord, s'il observe un état de l'objet $o_i$ à $t$, il sait le vectoriser dans $\mathbb{R}^v$ via une fonction idoine. Il calcule alors $\Psi(E_{it})$. De même, s'il observe les changements $c_{ij}^t$, il dispose d'une fonction $\Phi_{it} : X^I \rightarrow \mathbb{R}^v$ (qui peut dépendre de $i$ et $t$). Il calcule alors $R_{it} = \Phi_{it}((c_{ij}^t)_{j \in I}) \in \mathbb{R}^v$. Par exemple, si $I$ est fini, les $c_{ij}^t$ sont des vecteurs et $\Phi_{it} = \sum_{j \in I}$ devient indépendant du temps et de l'objet. 
	\item il applique alors l'équation générale et trouve le $\Delta E_{it}$. Il peut appliquer une rétroaction sur la $M_{it}$ si son modèle l'exige. Enfin, il calcule le nouvel état. 
\end{enumerate}

\section{Graphe d'états}

Le modèle précédent a pour vocation de poser une logique d'états expliqués par des changements traités par un objet. Nous proposons ici une autre piste, basée sur un graphe dynamique. Par nature, un graphe est une structure finie ou dénombrable. Le principe est de prendre des nœuds qui reçoivent des événements (ce qu'on appelait changement) et qui les traitent. Ces événements transitent via des liens qui les propagent avec une certaine latence. On note $\mathcal{E}$ l'ensemble des ensembles finis d'événements. Très pratiquement, on utilisera en fait $e \in \mathcal{E}$ pour dire "un ensemble fini d'événements"

\begin{itemize}
	\item les nœuds sont nos objets $o_i$, ils une durée de vie : hors de cet ensemble, ils ne traitent pas de message. Chaque élément actif est une fonction $F_i : \mathcal{E} \rightarrow \mathcal{E}$
	\item les liens sont orientés et valués par $\delta t > 0$. Leur source et leur destination sont donc des nœuds. Ils garantissent que si la source produit les événements $e \in \mathcal{E}$ à $t$, alors la destination les reçoit au plus tôt à $t+dt$ (où $t$ est le temps du lien, pas nécessairement celui de sa source ou de sa destination). Un lien a lui aussi une durée de vie : il peut ne pas transmettre d'événement à un moment donné. 
	\item les nœuds peuvent avoir un état initial. Dès lors, le graphe les ajoute avec cet état initial et ils sont modifiés au fur et à mesure des événements produits. 
\end{itemize}
$$\text{}$$
En terme d'implémentation, la description très large du nœud permet d'avoir des états conservés qui déterminent la gestion des événements. Par exemple, on prend un réseau social avec les liens de suivre un compte. Dès lors, les événements sont interprétés comme des messages changeant l'état émotionnel d'un individu. Cet individu va alors suivre des comptes en fonction de son état courant. 

$$\text{}$$
On peut alors exploiter le modèle précédent de la manière suivante: 
\begin{tabular}{|l|l|l|}
	\hline 
	Version modèle & Version graphe \\
	\hline 
	Objet & Nœud \\
	\hline 
	Espace des objets & Graphe des objets \\
	\hline 
	État dans le modèle & État dans une implémentation \\
	\hline 
	Variation d'état & Variation d'état (iso) \\
	\hline 
	Résultante des changements & Réaction aux événements \\
	\hline 
	Matrice des états & Sensibilité aux événements \\
	\hline
\end{tabular}
$$\text{}$$

Bien sûr, d'autres formes plus évoluées sont possibles: 
\begin{itemize}
	\item des liens qui sont enrichis de certaines caractéristiques (usure, par exemple). On peut les affubler d'un nombre d'événements maximum par unité de temps, ou une usure qui les désactive sous certaines conditions. 
	\item des liens qui filtrent les événements parce qu'ils ne sont pas sensibles à cette forme d'événement. Par exemple, un champ gravitationnel peut être insensible à toute forme de force électromagnétique.
\end{itemize}

\end{document}