\documentclass{article}

\title{Vers un moteur de simulation sociale}
\author{zefrenchwan}
\date{version du \today{}}

\usepackage[utf8]{inputenc}
\usepackage[T1]{fontenc}
\usepackage[french]{babel}
\usepackage{amsfonts,amsmath,amssymb}
\usepackage{float}

\begin{document}
\maketitle
\abstract{Nous avons chacun un point de vue sur notre système social, et nous passons notre temps à anticiper sa dynamique : ce qui sera fait, par qui, à quelle échéance. Cette capacité nous permet d'ajuster nos comportements au fur et à mesure, que ce soit tactiquement dans une situation donnée ou pour notre meilleur intérêt à plus long terme. La modélisation formelle d'un système social n'a rien de nouveau : elle est courante en politique (sondages, campagnes électorales), en marketing digital (maximiser l'impact d'une campagne) pour ne citer que celles-ci. 	Cet article propose un outil de modélisation applicable sur les systèmes sociaux par analogie avec les  systèmes physiques. Son utilisation consiste à définir des axes d'analyse (rancoeur, colère, espoir par exemple) et à formaliser un espace d'états pour observer au mieux les acteurs sociaux concernés. En distinguant les contraintes du système social des réactions propres aux acteurs, nous définissons une trajectoire d'états qui réconcilie aspirations individuelles et contraintes systémiques. Nous proposons également une formalisation des mécanismes d'apprentissage de chaque acteur. Mais soyons honnêtes : nous proposons donc un outil que nous espérons utile tout en sachant pertinemment que le social ne se laisse pas enfermer dans des équations, si complexes soient elles. }
\tableofcontents
\clearpage

\section{Vers une modélisation des systèmes sociaux}

Dans sa définition la plus générale, un système social est un ensemble d'entités évoluant au sein d'une structure d'interactions. Une entité peut être un individu, un groupe ou une institution, caractérisée par une identité stable et une durée de vie finie. Contrairement à un objet physique inerte, l'entité sociale est un acteur : elle comprend le monde de manière partielle et agit en fonction de buts spécifiques. Dès lors, chaque acteur met en œuvre des stratégies. Ce qu'il comprend du système, notamment des autres acteurs, est fonction des faits qu'il connait mais aussi, et surtout, de son point de vue sur le système. Un même événement peut donc déclencher des réactions opposées selon le prisme idéologique de l'acteur qui le reçoit. Si l'on isole les concepts centraux, on a donc ceci : 

\begin{table}[H]
\center
\begin{tabular}{|l|l|}
\hline 
Nom & Description \\
\hline 
Acteur & Toute entité active (qui agit) et réflexive (qui pense) \\
\hline 
Lien & Toute forme d'interaction ou de contrainte \\
\hline 
Système & Cadre dans lequel les acteurs s'organisent \\
 & en fonction des liens. \\
\hline 
Fait & Ce qui est observable ou objectivement vrai \\
\hline 
Vue & Point de vue sur une partie du système \\
 & (par opposition au fait qui est objectif) \\
\hline 
But & Ce qu'un acteur veut obtenir, être ou réaliser \\
\hline 
Stratégie & Actions ou solutions trouvées par un acteur pour \\
 & réaliser un but en propre ou au sein d'une alliance \\
\hline 
\end{tabular}
\caption{Principaux concepts d'un modèle social}
\end{table}

Le concept clé que ce document veut mettre en avant est le suivant : \textbf{les vues définissent le comportement des acteurs, les comportements des acteurs modifient le système, le système change les vues des acteurs}. 
Donc, un système social est en rétroaction permanente: 
\begin{enumerate}
	\item \textbf{Perception} : Les vues induisent des tensions.
	\item \textbf{Action} : Les tensions déclenchent des actions qui modifient le système (création ou rupture de liens).
	\item \textbf{Réaction} : La nouvelle structure du système modifie en retour la perception des acteurs. 
\end{enumerate}

$$\text{}$$
C'est tout ce qui fait la difficulté d'un modèle réellement pertinent : 
\begin{itemize}
	\item les acteurs communiquent rarement leurs vue et leur stratégie de manière exhaustive et transparente. Il faut donc l'approximer en observant les actes, ou l'estimer par rapport à leur logique de maximisation de leur propre intérêt (tel parti a un intérêt à une alliance pour placer ses candidats à telle élection). 
	\item les boucles de rétroaction permanentes amènent à des équations que l'on ne peut résoudre que sur des temps courts du fait de leur sensibilité aux conditions initiales. A supposer bien sûr que le système reste dans une configuration connue et mathématisable. 
	\item les événements inattendus ou difficiles à anticiper, les fameux cygnes noirs, sont une évidence : on sait qu'on en aura.
\end{itemize}

Aussi, soyons modestes, et disons le tout de suite : le social ne se laisse pas docilement prédire ainsi. Il faut donc distinguer plusieurs objectifs de précision en fonction de ce que l'observateur sait d'un système: 

\begin{enumerate}
	\item la prédiction définit de manière unique l'état suivant, ce qu'on va observer à coup sûr. 
	\item la prédiction fournit une distribution de probabilités, ou en tous cas la possibilité de décrire les prochains états en affirmant que celui ci est plus probable que celui là.
	\item la prédiction donne un ensemble des possibles, qui est une façon surtout de dire que les autres sont absolument impossibles. On restreint donc l'ensemble des possibles. 
\end{enumerate}

\begin{table}[H]
\center
\begin{tabular}{|l|l|}
\hline 
Niveau de prédiction & Description \\
\hline 
Déterministe & L'état suivant est connu \\
 & au fil du temps \\
\hline 
Probabiliste &  Les états suivants sont évaluables \\
 & suivant une distribution de probabilités \\
\hline 
Prospectif & On exclut les états impossibles \\
 & et on fournit des tendances à long terme \\
 \hline 
\end{tabular}
\caption{Niveaux de prédiction possibles}
\end{table}

\section{Modèle de gestion d'états d'un système}

Soyons très concret : un observateur est en mesure d'observer un système et de capter des événements. Il applique un modèle sur ce qu'il observe. Il y a bien alors deux systèmes : ce qui existe en soi, et ce qui est modélisé. Dès lors, notons $(o_i)_{i \in I}$ les objets réels, observables (avec en général $I$ fini ou dénombrable). 


Prenons un exemple des plus classiques, la mécanique Newtonienne : 
\begin{enumerate}
	\item on choisit un repère, qui est la base de la description de l'observateur. On ajoute alors chaque élément dans ce repère avec un état initial $E_{i0}$. 
	\item on fixe $t$ et on veut comprendre ou prédire l'observation de chaque élément du système en $t > 0$. En pratique, on aurait une version discrétisée des équations de la mécanique : au lieu de $dt$ instantané, on serait sur un pas de temps $\Delta t$. 
	\item on calcule la résultante $\mathcal{R}_{it}$ qui s'applique à l'objet $i \in I$ au temps $t$. Il s'agit de synthétiser les interactions de chaque $o_j$ ($j \in I$, $j \neq i$) en un vecteur, puis d'en calculer la somme. De manière plus générale, $\mathcal{R}_{it}$ est la résultante des changements qu'il subit du système observé.
	\item On applique l'équation $\mathcal{R}_{it} dt = d\vec{P}_i$ reliant la résultante à la quantité de mouvement $P_i$ de l'objet d'indice $i$. C'est là que le modèle transforme une cause extérieure en changement d'état (ici une variation de quantité de mouvement). Les invariants du modèle, à savoir le lien entre quantité de mouvement et position, permettent de conclure. On en déduit une variation de vitesse, puis un ajustement de la position, donc un nouvel état (dans l'espace des phases) $E_{it}$. 
\end{enumerate}  
$$\text{ }$$
Au delà des équations, c'est la méthode qui nous intéresse. Notre idée est d'extrapoler un modèle social à partir d'équations physiques. Nous distinguerons dans la suite les \textbf{acteurs sociaux} (ou simplement acteurs) qui sont spécifiquement dans le domaine social, des \textbf{objets} qui peuvent être des objets physiques comme des acteurs sociaux. Ceci posé, reprenons les notions clés que la physique nous apporte. Il nous faudra en particulier revenir sur: 
\begin{enumerate}
	\item le choix des axes d'analyse pour comprendre comment passer à un espace d'états. Le but est de s'affranchir des spécificités du problème initial pour le poser en terme de pertinence sur un certain nombre d'axes
	\item les variations d'états locales, pour passer de l'état à $t$ à celui pour le temps $t + dt$. Dans le cas physique, seule la résultante s'applique. Il nous faudra enrichir cette interprétation
	\item la trajectoire dans un espace d'états, qui est une façon de reformuler comment un objet varie dans le temps suivant certains axes. 
\end{enumerate}

\subsection{La modélisation par états}

L'observateur utilise un espace vectoriel pour modéliser les états des objets $o_i$. Ce point seul mérite qu'on s'y attarde. Dans un système mécanique, la question principale reste la prédiction de la trajectoire d'un objet au fil du temps. L'espace vectoriel en question ne pose pas spécialement de problème $\mathbb{R}^2$ pour les mouvements en deux dimensions, et $\mathbb{R}^3$ pour les mouvements en 3D (par définition). Dans un système social, définir ces axes d'analyse est une question tout à fait cruciale : des axes évidents donnent des informations triviales, et de trop nombreux axes induisent un temps de calcul rendant inopérantes les simulations. Mais dès que ces $v$ axes sont connus et "indépendants", ils induisent une base d'un espace vectoriel de dimension $v$. Donnons des exemples concrets: 
\begin{itemize}
	\item opinions politiques : si un pays donné a $v$ partis en lice pour une élection politique à venir, il suffit d'attribuer à chaque axe un parti politique. Le nombre sur l'axe est donc l'adhésion aux idées du parti
	\item marketing digital : si l'on segmente une offre sur $v$ produits, l'intérêt pour le produit devient un axe évident. 
	\item physique : les objets sont décrits suivant $v$ axes. Par exemple, dans l'espace des phases, $v$ est le nombre de degrés de liberté qui définissent comment le système peut évoluer
\end{itemize}


On ne se place pas dans l'espace réel dans lequel les objets évoluent, mais bien dans l'espace des états que peuvent prendre les objets du système. Une fois cet espace fixé, disons $\mathbb{R}^v$, on positionne chaque objet $o_i$ à sa position initiale, et l'on s'intéresse à son évolution au fil du temps. Autrement dit, toute la question est de prédire avec plus ou moins de certitude la position de l'objet $o_i$ dans l'espace d'état, donc de trouver $E_{i,t+dt}$ en fonction des états précédents, idéalement uniquement à partir de $E_{it}$ seulement. En physique, la connaissance de l'état courant suffit (en tous cas en mécanique classique). Et dans ce cas, sans perte de généralité, il suffit de trouver $E_{i,t+dt} - E_{it} = dE_{it}$. 


Pourquoi introduire un tel artifice mathématique ? D'abord, parce que les acteurs dans un même état agissent souvent de manière très proche. Il n'est pas nécessaire de traiter les évolutions au cas par cas, mais de prendre des groupes similaires et de regarder l'évolution de leurs états. Cela peut surprendre a priori. C'est pourtant une approximation fréquente en diplomatie : la connaissance de l'histoire permet de définir des tendances par analogie. Ensuite, l'outil est puissant pour modéliser notre propre incertitude. Si nous revenons aux limites de prédiction d'un modèle, on peut établir une correspondance mathématique : 
\begin{enumerate}
	\item dans le cas \textbf{déterministe}, $dE_{it}$ est parfaitement connu
	\item dans le cas \textbf{probabiliste}, $dE_{it}$ suit une loi de probabilité
	\item dans le cas \textbf{prospectif}, $dE_{it}$ forme un ensemble. En pratique, on discrétise le temps pour avoir, disons à $t+\Delta t$, un ensemble qui est celui des possibles. Par exemple un voisinage de rayon $r_t$ par rapport à la position à $E_{it}$. 
\end{enumerate}

\subsection{Modéliser les variations locales d'états}

Un objet $o_i$ subit de la part des autres objets $(o_j)_{j \in I, j \ne i}$ un certain nombre d'interactions. A un temps $t$ fixé, l'observateur isole les changements $c_{ij}^t$ qui vont avoir un impact sur l'état de $o_i$ causé par l'objet $j \ne i$. Il calcule alors $\mathcal{R}_{it}$ la résultante en fonction des $c_{ij}^t$. Il n'est pas nécessaire d'ailleurs que les $c_{ij}^t$ soient pris dans le même espace vectoriel, il suffit de savoir calculer $\mathcal{R}_{it} = \Psi( (c_{ij}^t)_{j \in I, j \ne i})$. C'est un élément essentiel de la modélisation : \textbf{il suffit de savoir calculer la résultante comme un vecteur, la nature exacte des $c_{ij}^t$ n'a pas d'importance pour le modèle lui même}. 


On interprète alors la résultante comme le changement extérieur qui s'applique sur l'objet $o_i$ à $t$. Voici des exemples de résultantes: 
\begin{table}[H]
	\center
	\begin{tabular}{|l|l|}
		\hline 
		Contexte & Interprétation de la résultante \\
		\hline 
		Marketing & Publicités captées par l'acteur \\
		& sur une période courte fixée \\
		\hline 
		Réseau social & Messages captés par l'acteur \\
		& lors de sa session de navigation \\
		\hline 
		Politique & Informations délivrées lors de l'événement \\
		& (meeting politique, émission spécifique ) \\
		\hline
	\end{tabular}
	\caption{En fonction du contexte, exemples de résultantes}
\end{table}


Le cas physique est simple : $\mathcal{R}_{it} dt = dE_{it}$ pour tout $t$ et tout $i$. La connaissance des résultantes et de la position initiale de chaque objet suffit à supprimer toute incertitude (en théorie). De plus, sans résultante, la variation d'état est nulle, et l'objet continue sur une ligne droite à l'infini à vitesse constante. Si l'équation physique a le mérite de la simplicité, elle ne s'applique évidemment pas au social : 
\begin{itemize}
	\item en l'absence de pression extérieure, il n'y aurait pas de variation d'état. Ce n'est pas réaliste puisque les acteurs sociaux ont en général des buts et des stratégies. Nous allons introduire un terme additionnel, $\mathcal{B}_{it}$ qui est ce que veut faire l'acteur en l'absence de contrainte. On aurait alors $dE_{it} = \mathcal{B}_{it}dt$ en l'absence de toute action extérieure, c'est à dire que chaque acteur suit son plan si rien ne le change.
	\item un acteur social est sensible à certains éléments plus qu'à d'autres. Par exemple, dans un espace d'états politiques, un acteur d'un parti sera beaucoup plus sensible aux arguments des partis proches qu'à ceux des partis qu'il déteste ou néglige. Si donc un acteur social reçoit une résultante $\mathcal{R}_{it}$, sa réaction doit tenir compte d'une sensibilité $\mathcal{S}_{it}$. C'est formellement une matrice carrée de taille $v \times v$.  Pour la comprendre, sur la ligne $n$ et la colonne $p$, on trouve l'impact qu'a l'axe $p$ sur l'axe $n$. Par exemple, si l'on prend une analyse de crise suivant les axes de rationalité (1), colère (2), peur (3) et espoir (4), le coefficient $s_{2,3}$ modélise l'impact de la peur sur la colère. Sur la diagonale ($n = p)$, on trouve l'impact de l'axe lui même (par exemple la pure colère). Si l'on peut exclure toute interdépendance des axes les uns avec les autres, il suffit de prendre $0$ sur tous les termes hors diagonale. De manière générale, des termes proches de $0$ marquent une indifférence, des termes élevés montrent une énorme sensibilité.
\end{itemize}
$$\text{ }$$
Notons bien que cette formulation ne dépend que des acteurs et leur interaction. En sociologie, on parle d'\textbf{agency} : ce que ferait un acteur social de manière autonome, sans contrainte de la société. Notre formulation devient : 
$$\mathcal{A}_{it} = \mathcal{S}_{it}\mathcal{R}_{it} + \mathcal{B}_{it}$$
Avec $\mathcal{A}_{it}$ la réaction de l'acteur quand il est soumis à une pression locale $\mathcal{R}_{it}$ tout en suivant son but $\mathcal{B}_{it}$. 

\subsection{Modéliser les trajectoires globales des acteurs}

L'espace des états n'est pas un outil de modélisation passif. Dans un contexte social fixé, on a en général peu d'acteurs sociaux plongés dans un cadre mondial. On peut donc envisager un cadre invariant et fixé subi par les acteurs sociaux en première approximation. Or, ce cadre ne permet pas a priori la transition d'un état à un autre sans opposer une friction ou au contraire devenir la suite logique. Par exemple, si l'on modélise un paysage politique, passer d'un parti de droite modéré à une extrême gauche n'est pas chose aisée, il y a une opposition logique, une grande distance à franchir. En terme de modélisation, nous appliquons le principe suivant : \textbf{l'acteur propose, le système dispose}. Il faut donc inclure un terme correctif : 
\begin{itemize}
	\item qui modélise une forme de saturation. On s'attend à passer d'un état à un de ses voisinages sur un temps court, pas de sauter magiquement d'un état à un état très éloigné sur un temps très court. Ainsi, sur des variations raisonnables $dE_{it}$ est utilisé, mais si $dE_{it}$ prend des valeurs trop extrêmes, la correction ramène $dE_{it}$ à une valeur raisonnable. Très concrètement, une colère sociale, une envie d'acquérir un produit donné, ou l'appréciation d'un personnage public, ne peuvent pas devenir infinis.  
	\item qui exclut ou en tous cas limite des transitions sur certains axes. C'est en quelque sorte le contexte local, ce que le système impose à tous les acteurs qui seraient dans cet état. Par exemple, l'accès à un emploi prestigieux peut être limité à certaines catégories sociales, toute chose égale par ailleurs. Cette limitation sociétale serait modélisée par une correction s'appliquant à tout $dE_{it}$ si l'axe spécifique de la catégorie sociale ne correspond pas aux catégories retenues pour l'accès au poste en question. On comprend alors que ce n'est pas l'acteur spécifiquement qui est concerné, mais c'est bien son état qui a créé cette différence.  
\end{itemize}
$$\text{}$$
Le point clé est que \textbf{l'espace ne contient pas seulement des positions, il contient les règles de déplacement valides pour chaque position}. C'est donc un espace fonctionnel. Mais plutôt que de pousser encore plus loin le formalisme, donnons une version algorithmique. 


On fixe $t$ et on suit une boucle pour chaque acteur (pour tout $i \in I$) :
\begin{enumerate}
	\item on calcule $\mathcal{R}_{it}$ et on récupère $\mathcal{S}_{it}$ et $\mathcal{B}_{it}$. La résultante dépend des acteurs (via les $c_{ij}^t $). La sensibilité et le but sont des propriétés intrinsèques dynamiques de l'acteur. 
	\item on calcule ce que ferait l'acteur s'il n'était soumis à aucune contrainte, donc son "agency" : $\mathcal{A}_{it} = \mathcal{S}_{it} \mathcal{R}_{it} + \mathcal{B}_{it}$
	\item à ce stade, on a à la fois l'état courant de l'acteur $E_{it}$, et son effort d'adaptation $A_{it}$. Cet état courant détermine une réaction du système qui s'applique à tous les acteurs ayant ce même état. Le fait que l'on boucle sur les acteurs ne doit pas faire oublier que c'est leur état et pas l'acteur qui induit une limitation $\mathbb{L}(E_{it})$ des états possibles.  Pour la position courante $E_{it}$, la position future corrigée par le système devient $\mathbb{L}(E_{it})A_{it}$. 
	\item enfin, on applique une correction mathématique $\sigma$ pour éviter les valeurs aberrantes. Formellement, $\sigma$ est une fonction de $\mathbb{R}^v$ dans $\mathbb{R}^v$. En pratique, c'est une sigmoïde au sens mathématique : elle est à peu près linéaire pour des valeurs raisonnables, et écrase les valeurs trop extrêmes vers un résultat plausible. 
\end{enumerate}

Finalement, la formule complète devient : 
$$ E_{i,t+dt} = E_{it} + \sigma (\mathbb{L}(E_{it}) (\mathcal{S}_{it} \mathcal{R}_{it} + \mathcal{B}_{it})) dt$$
Le tableau suivant récapitule tous les symboles utilisés et leur interprétation : 
\begin{table}[H]
\center
\begin{tabular}{|l|l|}
\hline 
Terme & Signification \\
\hline 
$v$ & Nombres d'axes choisis par l'observateur \\
 & pour décrire fidèlement les acteurs observés \\
\hline 
$\mathbb{R}^v$ & espace des états possibles \\
 & (un nombre par axe ) \\
\hline 
$E_{i,t+dt}$ & Prochain état de l'acteur prévu par le modèle \\
\hline 
$E_{it}$ & État de l'acteur (observé ou simulé) \\
\hline 
$dE_{it}$ & Mouvement réel de l'acteur corrigé par le système \\
 & (et normalisé par $\sigma$) \\
\hline 
$dt$ & La durée durant laquelle s'applique le changement \\
& (la simulation numérique prend un pas de temps $\Delta t$) \\ 
\hline 
$\sigma$ & correction mathématique pour exclure \\
 & les valeurs extrêmes ou aberrantes \\
\hline 
$\mathbb{L}$ & Réaction du système aux mouvements des acteurs \\
 & (s'applique à tous les acteurs de même état) \\
\hline 
$\mathcal{R}_{it}$ & Pression subie par l'acteur exercée par les acteurs voisins \\
\hline 
$\mathcal{S}_{it}$ & Sensibilité de l'acteur aux pressions extérieures \\
\hline 
$\mathcal{B}_{it}$ & Buts que l'acteur se donne sur son développement \\
\hline 
\end{tabular}
\caption{Interprétation contextualisée des termes de l'équation}
\end{table}

\subsection{Les mécanismes de rétroaction}

Un acteur ne subit pas les changements sans réaction : il fait preuve d'adaptation. Au fur et à mesure qu'il évolue, ses réactions (au sens de $dE_{it}$) vont tenir compte de l'importance à traiter telle pression extérieure avec plus ou moins d'urgence ou d'importante. Formellement, c'est sa matrice de sensibilité qui va varier au fil du temps. Ainsi, si l'on inclut des mécanismes de rétroaction de l'acteur, ils portent sur $dS_{it}$. Mais de quoi dépend t'elle ? De la variation d'état $dE_{it}$ bien évidemment, mais aussi de $\mathcal{R}_it$ par construction. Autrement dit, l'apprentissage individuel s'écrit $$dS_{it} = \mathbb{A}(dE_{it},\mathcal{R}_{it})dt$$
avec $\mathbb{A}$ une fonction d'apprentissage. Il existe plusieurs façons de la modéliser (règles d'Oja ou de Hebb) mais l'on peut écrire, par exemple : 
$$\displaystyle { S_{i,t+dt} = S_{it} + \eta ( dE_{it} \otimes \mathcal{R}_{it} ) dt } $$
Soient $dS_{it} = [\sigma_{np}]_{1 \le n,p \le v}$, $\mathcal{R}_{it} = [r_n]_{1 \le n \le v}$ le vecteur colonne résultante et $dE_{it} = [\epsilon_n]_{1 \le n \le v}$ le vecteur colonne variation d'état. Alors, pour $n$ et $p$ fixés, il vient 
$$\sigma_{np} = \eta \times \epsilon_n  \times r_p \times dt$$
Fixons $n$ et $p$ pour traiter valeur à valeur. La formule s'explique ainsi : 
\begin{enumerate}
	\item la sensibilité $s_{np}$ de l'acteur sur la ligne $n$ et la colonne $p$ s'interprète par l'action de l'axe $p$ sur l'axe $n$. Si $n = p$, c'est simplement la valeur sur l'axe en question. 
	\item sa variation $\sigma_{np}$ est ce qu'on va ajouter à $s_{np}$ pour obtenir la nouvelle matrice de sensibilité
	\item une capacité globale d'apprentissage $\eta$. C'est un réel compris entre 0 et 1. Plus il est élevé, plus l'acteur va changer son comportement et être influencé par l'événement. Dans un système physique sans rétroaction, on aura $\eta = 0$. 
	\item $\epsilon_n$ qui est la variation d'état sur l'axe $n$. C'est bien une variation. En effet, si l'état ne change pas, il n'y a pas de raison de modifier la sensibilité aux événements, donc d'apprendre. 
	\item $r_p$ qui est la résultante des événements sur l'axe $p$. Plus elle est forte sur cet axe, plus grand est l'impact sur $\sigma_{np}$
\end{enumerate}
$$\text{ }$$
La dynamique de la société est un sujet qui mériterait un article à part et mobiliserait des connaissances que nous n'avons pas, en toute honnêteté. Nous ne pouvons donc qu'esquisser une idée : les variations du système social au temps $t$ sont liées à la série $(E_{it})_{i \in I}$. Autrement dit, il faudrait tenir compte de la "moyenne" des états des acteurs, et de la dispersion (écart type) des états des acteurs du système. Si celui ci devient trop grand, il y aurait peut être une organisation en sous systèmes plus homogènes ("small worlds"). 
\end{document}
